\documentclass{letter}
\usepackage{url}
\usepackage{hyperref}
\usepackage{graphicx}
\usepackage{xcolor}

%%%%%%%%%%%%%%%%%%%%%%%%%%%%%%%%%%%%%%%%%%%%%%%%%%%%%%%%%%%%%%%%%%%%%%%%%%%%
\begin{document}
\signature{Will Farr}

\address{Will M. Farr\\School of Physics and Astronomy\\University of Birmingham\\Birmingham\\B15 2TT\\United Kingdom\\\href{mailto:w.farr@bham.ac.uk}{\texttt{w.farr@bham.ac.uk}}}

\begin{letter}
{Fred Rasio \\
\href{mailto:ApJLFred@northwestern.edu}{\texttt{ApJLFred@northwestern.edu}}}

\opening{Dear Dr. Rasio:}

Please find enclosed a submission for an ApJ Letter titled ``Aldebaran~b's
temperate past uncovered in planet search data''.

Aldebaran ($\alpha$~Tauri, the Eye of the Bull) is a nearby first-magnitude red
giant star. Since the identification of the planet candidate Aldebaran~b in~1993
in the first modern radial velocity (RV) surveys, the existence of a massive
planet orbiting the star has been firmly established with a further two decades
of observations.

One astronomer's noise is another astronomer's data: using new data processing
techniques, we have found that some of the ``noise'' in these archival
observations is actually the signal of stellar oscillations. We confirmed our
result with new observations. The asteroseismic oscillations allow us to infer
much more accurate stellar parameters and therefore orbital parameters for the
planet; we constrain the star's age, measure its mass with a factor of two
better precision than had previously been possible (to 6\%), and consequently
obtain the mass of the giant planet to a precision of 12\%.

Our technique will be widely applicable to legacy and future datasets, allowing
both asteroseismology with RV, and also searches for lower-mass companions to
giants than was previously possible.  There are many targets with \emph{already
existing} archival data amenable to this technique, and it will also be
applicable to upcoming and ongoing large surveys (Gaia, LSST, etc).

We have also independently confirmed our results with new photometry from the
\emph{Kepler}-2 (K2) mission, again using a novel algorithm to extract a precise
light curve of the brightest star ever observed by \emph{Kepler}. The K2 light
curve shows very clear oscillations that are in agreement with the frequency we
infer from RV.

A neat and surprising result of this work is that while the planet has a hellish
temperature today well upwards of a thousand Kelvin, when Aldebaran was on the
main sequence, the planet (and more importantly any of its moons) would have
been subject to similar incident starlight as the Earth receives from the Sun.
We posit that this is the first clear example of a formerly-habitable world
rendered uninhabitable by stellar evolution.

We would like to have this submission considered for a linked Letter+Supplement;
we have formatted it as one document, but the appendices are intended for the
Supplement.  It is not clear to us whether there are length limits still applied
to letters; the main body is currently about 4000 words, contains six figures
(two of which could be merged if necessary), and sixty references (this could be
reduced by moving some references to the Supplement if necessary).

Thank you for considering our submission. Please do not hesitate to contact me
with any questions. 

\closing{Sincerely,}

\end{letter}

\end{document}
