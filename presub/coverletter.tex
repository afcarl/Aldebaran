% Cover letter using letter.cls
\documentclass[11pt]{letter} % Uses 10pt
%\usepackage{helvetica} % uses helvetica postscript font (download helvetica.sty)
%\usepackage{newcent}   % uses new century schoolbook postscript font 
% the following commands control the margins:

\topmargin=-1.5in    % Make letterhead start about 1 inch from top of page 
\textheight=11.75in    % text height can be bigger for a longer letter
\oddsidemargin=0pt   % leftmargin is 1 inch
\textwidth=6.5in     % textwidth of 6.5in leaves 1 inch for right margin
\usepackage{url}
\usepackage{hyperref}
\hypersetup{
    colorlinks=false,
    pdfborder={0 0 0},
}
\usepackage{graphicx}
\usepackage[strict]{changepage}
\usepackage{xcolor}

%--------------------------------------------------------------
% A few colors for hyperlinks.
%--------------------------------------------------------------
\definecolor{plum}{rgb}{0.36078, 0.20784, 0.4}
\definecolor{chameleon}{rgb}{0.30588, 0.60392, 0.023529}
\definecolor{cornflower}{rgb}{0.12549, 0.29020, 0.52941}
\definecolor{scarlet}{rgb}{0.8, 0, 0}
\definecolor{brick}{rgb}{0.64314, 0, 0}




\newcommand{\email}[1]{\href{mailto:#1}{\tt \textcolor{cornflower}{#1}}}

%%%%%%%%%%%%%%%%%%%%%%%%%%%%%%%%%%%%%%%%%%%%%%%%%%%%%%%%%%%%%%%%%%%%%%%%%%%%
\begin{document}


%
\longindentation=0pt                       % needed to get closing flush left
\let\raggedleft\raggedright                % needed to get date flush left

\nopagebreak  
\begin{letter}
{Leslie Sage \\
l.sage@us.nature.com \\}
      
{
\leftskip = 0pt plus 1 fill
\rightskip = 0pt
\parindent 0pt
%\obeylines
%
%
{{\bf Dr Will M. Farr}} \\
School of Physics and Astronomy\\
University of Birmingham\\
Birmingham, B15 2TT\\
United Kingdom\\
 \url{w.farr@bham.ac.uk} \\
 
}


\begin{minipage}[c]{5in}\vskip-4.25cm
\begin{flushleft}
	\begin{minipage}[c]{3cm}
		\begin{flushleft}
			\includegraphics*[width=8cm]{birmingham_logo.png}%
		\end{flushleft}
	\end{minipage}
\end{flushleft}
\end{minipage}

\opening{Dear Leslie,} 
 
\noindent I submit for your consideration a Letter on the ``Detection of Oscillations in Aldebaran with Ground-Based Observations''.

Aldebaran ($\alpha$~Tauri, the Eye of the Bull) is a nearby first-magnitude red giant star. Since the identification of the planet candidate Aldebaran~b in~1993 in the first modern radial velocity (RV) surveys, the existence of a massive planet orbiting the star has been firmly established with a further two decades of observations. 

One astronomer's noise is another astronomer's data: we have found that the noise in these archival observations is actually the signal of stellar oscillations. With radical new data processing techniques, we have re-analysed historical RV data and confirmed our result with new observations. Using our Gaussian Process model we detect the signal of stellar oscillations, missed by previous authors, and consequently infer much more accurate orbital parameters for the planet. This technique will be widely applicable to legacy and future datasets, allowing both asteroseismology with RV, and also searches for lower-mass companions to giants than was previously possible.

We have also independently confirmed our results with new photometry from the \emph{Kepler}-2 (K2) mission, again using a highly novel algorithm to extract a precise light curve of the brightest star ever observed by \emph{Kepler}. The K2 light curve shows very clear oscillations that are in perfect agreement with the frequency we infer from RV. 

With asteroseismology we constrain the star's age for the first time, measure its mass with a factor of two better precision than had previously been possible (to 6\%), and consequently obtain the mass of the giant planet to a precision of 12\%.

A neat and surprising result of this work is that while the planet has a hellish temperature today well upwards of a thousand Kelvin, when Aldebaran was on the main sequence, the planet (and more importantly any of its moons) would have been subject to similar incident starlight as the Earth receives from the Sun. We posit that this is the first clear example of a formerly-habitable world rendered uninhabitable by stellar evolution.

Thank you for considering our submission. Please do not hesitate to contact me with any questions.\\

\closing{Yours sincerely, \\
\vspace{1.0cm}
Will Farr} 
  
%\encl{One paragraph Letter outline}					% Enclosures

\end{letter}
 
\end{document}